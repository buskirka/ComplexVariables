\documentclass{article}

\title{Homework 2016-04-14}
\author{Adam Buskirk}

\usepackage{amssymb,amsmath,amsthm}
\usepackage{tikz}
\usepackage[margin=1in]{geometry}

\newtheorem{theorem}[subsection]{Theorem}
\newtheorem{conjecture}[subsection]{Conjecture}
\newtheorem{lemma}[subsection]{Lemma}
\theoremstyle{definition}
\newtheorem{definition}[subsection]{Definition}

\allowdisplaybreaks[1]

\newcommand{\R}{\mathbb{R}}
\newcommand{\N}{\mathbb{N}}
\newcommand{\Q}{\mathbb{Q}}
\newcommand{\Z}{\mathbb{Z}}
\newcommand{\Co}{\mathbb{C}}
\newcommand{\p}[1]{\left(#1\right)}
\newcommand{\sq}[1]{\left[#1\right]}
\newcommand{\set}[1]{\left\{#1\right\}}
\newcommand{\abs}[1]{\left|#1\right|}
\newcommand{\norm}[1]{\left|\left|#1\right|\right|}
% \newcommand{\p}[1]{\left(#1\right)}

\begin{document}
\maketitle

\section{Problem 5, p.\ 206}
\[ f(z) = \frac{z+1}{z-1} \]
\subsection{Part a}
To obtain a Maclaurin series for this, first we must compute the derivatives.
\begin{align*}
f(z)
&= \frac{z+1}{z-1} \\
f'(z)
&= \frac{1}{z-1} + (-1) \frac{z+1}{(z-1)^2} \\
&= \frac{z-1}{(z-1)^2} - \frac{z+1}{(z-1)^2} \\
&= 2\frac{(-1)^1 \cdot 1!}{(z-1)^{1+1}}
\intertext{Suppose then that $f^{(n)}(z)$ is of the form $2 (-1)^n n! / (z-1)^{n+1}$,
as is above for $n=1$.}
f^{(n+1)}(z) 
&= (-n-1) \cdot 2 \frac{(-1)^n \cdot n!}{(z-1)^{n+2}}\\
&= 2 \frac{(-1)^{n+1} \cdot (n+1)!}{(z-1)^{n+2}} 
\intertext{which is also of that form. Thus, by induction, for $n \ge 1$,}
f^{(n)}(z)
&= 2 \frac{(-1)^n \cdot n!}{(z-1)^{n+1}}
\end{align*}
Thus, the coefficients $a_n$ of our Maclaurin series are $a_0=-1$ and
\[
a_n 
= \frac{f^{(n)}(0)}{n!}
= \frac{1}{n!} \cdot 2 \frac{(-1)^n \cdot (n!)}{(0-1)^{n+1}}
= -2
\]
Thus,
\[
f(z)
= -1 - 2 \sum_{n=1}^\infty z^n
\]
and since the analytic circle around this only ends at $R_0=1$, 
this Maclaurin series holds for the disk $|z|<1$.

\subsection{Part b}
For the Laurent series, 
\begin{align*}
\frac{z+1}{z-1}
&= \\
&= \cdots \\
&= 1+2\sum_{n=1}^\infty \frac{1}{z^n}
\end{align*}
since
\begin{align*}
a_n 
&= \frac{1}{2 \pi i} \int_C \frac{f(z)}{(z-0)^{n+1}} \;dz \\
&= \frac{1}{2 \pi i} \int_C \frac{\frac{z+1}{z-1}}{z^{n+1}} \;dz \\
&= \frac{1}{2 \pi i} \int_C \frac{z+1}{(z-1)z^{n+1}} \;dz \\
&= \frac{1}{2 \pi i} \int_C \frac{1}{(z-1) z^n} + \frac{1}{(z-1) z^{n+1}} \;dz
\end{align*}
I do not understand how to continue this but I also do not want to delay handing
in this assignment even further. Previous attempts based on re-evaluating the 
interior of those integrals as
\[ \frac{\frac{1}{z-1}}{z^n} \]
and applying the Cauchy integral formula lead to the conclusion that $a_n = 2$
or something of that sort, and this is evidently not the case based on the 
answer in the book; and no progress was made with $b_n$.

\section{Problem 1}
Evaluate \[ \int_C \frac{z+1}{z-1} \;dz \]
where $C$ is a positively oriented contour around $1$.

Expand $C$ to be the positively oriented circle contour 
of radius 2 around $0$, so that it lies
entirely within the domain of the Laurent series representation.

\begin{align*}
\int_C \frac{z+1}{z-1} \;dz 
&= \int_C 1+2 \sum_{n=1}^\infty \frac{1}{z^n} \;dz \\
&= \int_C 1 \;dz + 2 \sum_{n=1}^\infty \int_C \frac{1}{z^n} \;dz \\
&= 0 + 2\int_C \frac{1}{z} \;dz + 2 \sum_{n=1}^\infty \int_C \frac{1}{z^{n+1}} \;dz
\intertext{Utilizing the Cauchy integral formula, and specifically
example 2 on page 168,}
&= 0 + 2 (2 \pi i) +2 \sum_{n=1}^\infty 0 \\
&= 4 \pi i
\end{align*}

\section{Problem 2}
Find the Laurent series for 
\[ f(z)= \frac{2}{1-z^2} = \frac{1}{1-z} + \frac{1}{1+z} \]

We utilize example 4 in the text, on page 194.
\begin{align*}
f(z) &= \frac{1}{1-z} + \frac{1}{1+z} \\
&= \sum_{n=0}^\infty z^n + \sum_{n=0}^\infty (-z)^n \\
&= \sum_{n=0}^\infty z^n + \sum_{n=0}^\infty (-1)^n z^n \\
&= \sum_{n \in 2\N} z^n \tag{Note the change of index! We assume $0\in\N$.}\\
&= \sum_{n=0}^\infty z^{2n} \\
&= \sum_{n=0}^\infty (z^2)^n \\
&= 2 \frac{1}{1-z^2} \\
&= \frac{2}{1-z^2} \\
&= f(z)
\end{align*}
Thus, the Laurent series for $f(z)$ is $\sum_{n=0}^\infty z^{2n}$.
\end{document}
