\documentclass{article}

\usepackage{amsmath,amssymb,amsthm}
\usepackage[margin=1.25in]{geometry}

\author{Adam Buskirk}
\title{Complex Variables Section 57 Homework}

\begin{document}
\maketitle

\section{Problem 1}
\[ 
f(z)=\begin{cases}
z^{-1}\sin(z) & \text{if } x \neq 0 \\
1 & \text{if } x = 0
\end{cases}
\]

On $\mathbb{C} \setminus \{0\}$, 
\begin{align*}
f(z) 
&= z^{-1} \sum_{n=0}^\infty (-1)^n \frac{z^{2n+1}}{(2n+1)!} \\
&= \sum_{n=0}^\infty (-1)^n \frac{z^{2n}}{(2n+1)!} \\
&= \sum_{n=0}^\infty \frac{(-1)^n}{2n+1} \cdot \frac{z^{2n}}{(2n)!} \\
\end{align*}
which can be naturally extended to all of $\mathbb{C}$.
\section{Problem 2}
In the first derivative, the first term drops out.
\begin{align*}
f'(z)
&= \sum_{n=1}^\infty \frac{(-1)^n}{2n+1} \cdot \frac{z^{2n-1}}{(2n-1)!} \\
f^{(2)}(z)
&= \sum_{n=1}^\infty \frac{(-1)^n}{2n+1} \cdot \frac{z^{2n-2}}{(2n-2)!} \\
\intertext{In the third derivative, another term is forced to drop out.}
f^{(3)}(z)
&= \sum_{n=2}^\infty \frac{(-1)^n}{2n+1} \cdot \frac{z^{2n-3}}{(2n-3)!} \\
f^{(4)}(z)
&= \sum_{n=2}^\infty \frac{(-1)^n}{2n+1} \cdot \frac{z^{2n-4}}{(2n-4)!} \\
\end{align*}
Thus, 
$f^{(4)}(0) = (-1)^2 (2 (2) + 1)^{-1} (0!)^{-1} = 1 / 5 = 0.2$.



\end{document}
