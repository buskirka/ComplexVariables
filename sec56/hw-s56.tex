\documentclass{article}

\title{Complex Variables Section 56 Homework}
\author{Adam Buskirk}

\usepackage{amssymb,amsmath,amsthm}
\usepackage{tikz}
\usepackage[margin=1in]{geometry}

\newtheorem{theorem}[subsection]{Theorem}
\newtheorem{conjecture}[subsection]{Conjecture}
\newtheorem{lemma}[subsection]{Lemma}
\theoremstyle{definition}
\newtheorem{definition}[subsection]{Definition}

\newcommand{\R}{\mathbb{R}}
\newcommand{\N}{\mathbb{N}}
\newcommand{\Q}{\mathbb{Q}}
\newcommand{\Z}{\mathbb{Z}}
\newcommand{\Co}{\mathbb{C}}
\newcommand{\p}[1]{\left(#1\right)}
\newcommand{\sq}[1]{\left[#1\right]}
\newcommand{\set}[1]{\left\{#1\right\}}
\newcommand{\abs}[1]{\left|#1\right|}
\newcommand{\norm}[1]{\left|\left|#1\right|\right|}
% \newcommand{\p}[1]{\left(#1\right)}

\begin{document}
\maketitle

Page 188 : 3

Prove that if $\abs{z_n} \to 0$, then $z_n \to 0$.

Converge or diverge: $\sum_{k=1}^\infty (i/2)^k$, 
$\sum_{k=1}^\infty (1-i)^k$.

\section{Page 188, \#3}
\begin{theorem}
If $\lim_{n\to\infty} z_n = z$, then $\lim_{n \to \infty} \abs{z_n} = \abs{z}$.
\end{theorem}
\begin{proof}
Suppose $z_n \to z$. Then for any $\epsilon>0$, there exists $N$ such that if
$n \ge N$, then
\[
\abs{z_n - z} < \epsilon
\]
\[
\abs{\abs{z_n}-\abs{z}} \le \abs{z_n - z} < \epsilon
\]
Thus, for any $\epsilon$ there exists $N$ such that 
\[ \abs{\abs{z_n} - \abs{z}} < \epsilon \]
so $\abs{z_n} \to \abs{z}$.
\end{proof}

\section{Modulus convergence}
\begin{theorem}
If $|z_n| \to 0$, then $z_n \to 0$.
\end{theorem}
\begin{proof}
For an arbitrary $\epsilon$, 
there exists some $N$ such that
$n \ge N \implies |z_n - 0| = |z_n| < \epsilon$.
Then $\sqrt{x_n^2 + y_n^2} < \epsilon$, so
$x_n^2 + y_n^2 < \epsilon^2$
so $x_n^2 < \epsilon^2$ and $y_n^2 < \epsilon$, 
and thus $\abs{x_n-0} < \epsilon$ and 
$\abs{y_n-0} < \epsilon$. Since
$\epsilon$ was arbitrary, $x_n \to 0$ and $y_n \to 0$.
Thus $z_n = x_n + i y_n \to 0 + i 0 = 0$.
\end{proof}

\section{Converge or diverge}
\subsection{First one}
\[ 
\sum_{k=1}^\infty (i/2)^k
= i/2 - 1/4 - i/8 + 1/16 + \cdots
= \sum_{k=1}^\infty \p{-\frac{1}{4}}^k 
- i \frac{1}{2} \sum_{k=1}^\infty \p{-\frac{1}{4}}^{k-1}
\]
The summations work out individually, utilizing the ratio test, 
so the complex sums converge also.

\subsection{Second one}
This cannot converge, since the underlying sequence does not converge
to $0$; by the ratio test,
\[ \abs{\frac{(1-i)^{k+1}}{(1-i)^k}} = \abs{1-i} = \sqrt{2} > 1 \]
And since the sequence being summed does not converge to $0$, the summation
itself cannot converge.

\end{document}
