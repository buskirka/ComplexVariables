\documentclass{article}

\title{Complex Variables Final Exam}
\author{Adam Buskirk}

\usepackage{amssymb,amsmath,amsthm}
\usepackage{tikz}
\usepackage[margin=1in]{geometry}

\allowdisplaybreaks[1]

\newtheorem{theorem}[subsection]{Theorem}
\newtheorem{conjecture}[subsection]{Conjecture}
\newtheorem{lemma}[subsection]{Lemma}
\theoremstyle{definition}
\newtheorem{definition}[subsection]{Definition}

\newcommand{\R}{\mathbb{R}}
\newcommand{\N}{\mathbb{N}}
\newcommand{\Q}{\mathbb{Q}}
\newcommand{\Z}{\mathbb{Z}}
\newcommand{\Co}{\mathbb{C}}
\newcommand{\p}[1]{\left(#1\right)}
\newcommand{\sq}[1]{\left[#1\right]}
\newcommand{\set}[1]{\left\{#1\right\}}
\newcommand{\abs}[1]{\left|#1\right|}
\newcommand{\norm}[1]{\left|\left|#1\right|\right|}
% \newcommand{\p}[1]{\left(#1\right)}

\begin{document}
\maketitle

\section{Problem 1}
We decompose $S$ into four portions.
\begin{align*}
\int_S \bar{z} \;dz
&= 
  \int_0^2 x \;dx
+ \int_0^2 2+yi \;dy
+ \int_2^0 x+2i \;dx
+ \int_2^0 yi \;dy \\
&= 
  \int_0^2 x \;dx
+ 4+i\int_0^2 y \;dy
- \int_0^2 x \;dx-4i
- i\int_0^2 y \;dy \\
&= 
  4-0
+ 4+i(4-0)
- (4-0)-4i
- i(4-0) \\
&= 4+4+4i-4-4i-4i \\
&= 4-4i
\end{align*}

\section{Problem 2}
The value of this integral over the spiral-shaped contour may be simplified. 
Namely, $z^2$ is entire; thus we may define a loop homotopy of $C$
into $C'(t)=1+t$ by
\begin{align*}
H &: [1,2] \times [0,1] \to \Co  &
H(t,s) &\overset{\text{def}}{=} (1-s) \cdot t e^{2 \pi i t} + s \cdot t
\end{align*}
Thus,
\begin{align*}
\int_C z^2 \;dz
&= \int_{C'} z^2 \;dz \\
&= \int_1^2 x^2 \;dx \\
&= \left.\frac{x^3}{3}\right|_1^2 \\
&= \frac{2^3}{3} - \frac{1^3}{3} \\
&= \frac{8-1}{3} \\
&= \frac{7}{3}
\end{align*}
\section{Problem 3}
\begin{align*}
\int_C \frac{\cos 2z}{(z-\pi)^3} \;dz
&= \frac{2 \pi i}{2!} \sq{D_z^2 (z \mapsto \cos(2z))}(\pi) \\
&= \pi i \sq{D_z (z \mapsto -2\sin(2z))}(\pi) \\
&= \pi i \sq{z \mapsto -4 \cos(2z)}(\pi) \\
&= -4 \pi i \cos(2\pi) \\
&= -4 \pi i
\end{align*}
\section{Problem 4}
\section{Problem 5}
\section{Problem 6}
\section{Problem 7}
\section{Problem 8}
\section{Problem 9}
\section{Problem 10}
\section{Problem 11}
\begin{theorem}
Suppose that $f$ is an entire function satisfying 
\[
f(z+i)=f(z) \text{ and } f(z+1)=f(z) \text{ for all } z
\]
Then $f$ is constant.
\end{theorem}
\begin{proof}
Note that if $f$ is entire it is continuous. Thus it takes compact subsets of $\Co$ to
compact subsets of $\Co$. Consequently, 
the image of $[0,1]+[0,1]i$ under $f$, $f([0,1] + [0,1]i)$, is bounded; that is,
\[
\abs{f([0,1]+[0,1]i)} < B
\]
for some bound $B \in \R$.

But then if $f([n,n+1] + [m,m+1]i)$ is bounded, we also know that by the presuppositions
of the theorem, all its neighboring cells share the same bound:
\begin{align*}
\abs{f([n,n+1]+[m,m+1]i}&=\abs{f([n+1,n+2]+[m,m+1]i} < B \\
\abs{f([n,n+1]+[m,m+1]i}&=\abs{f([n-1,n]+[m,m+1]i} < B \\
\abs{f([n,n+1]+[m,m+1]i}&=\abs{f([n,n+1]+[m+1,m+2]i} < B \\
\abs{f([n,n+1]+[m,m+1]i}&=\abs{f([n,n+1]+[m-1,m]i} < B \\
\end{align*}
Since every point of $\Co$ lies in some integer unit square in the complex
plane, and all of these by induction using the equations above are bounded
in modulus by $B$, $f$ is bounded in modulus by $B$ over all of $\Co$.
Since $f$ is bounded
in modulus by $B$, by the maximum modulus principle, $f$ must be constant.
\end{proof}

\end{document}
