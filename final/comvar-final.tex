\documentclass{article}

\title{Complex Variables Final Exam}
\author{Adam Buskirk}

\usepackage{amssymb,amsmath,amsthm}
\usepackage{tikz}
\usepackage[margin=1in]{geometry}

\allowdisplaybreaks[1]

\newtheorem{theorem}[subsection]{Theorem}
\newtheorem{conjecture}[subsection]{Conjecture}
\newtheorem{lemma}[subsection]{Lemma}
\theoremstyle{definition}
\newtheorem{definition}[subsection]{Definition}

\newcommand{\R}{\mathbb{R}}
\newcommand{\N}{\mathbb{N}}
\newcommand{\Q}{\mathbb{Q}}
\newcommand{\Z}{\mathbb{Z}}
\newcommand{\Co}{\mathbb{C}}
\newcommand{\p}[1]{\left(#1\right)}
\newcommand{\sq}[1]{\left[#1\right]}
\newcommand{\set}[1]{\left\{#1\right\}}
\newcommand{\abs}[1]{\left|#1\right|}
\newcommand{\norm}[1]{\left|\left|#1\right|\right|}
% \newcommand{\p}[1]{\left(#1\right)}

\begin{document}
\maketitle

\section{Problem 1}
We decompose $S$ into four portions.
%\begin{align*}
%\int_S \bar{z} \;dz
%&= 
%  \int_0^2 (x) (1) \;dx
%+ \int_0^2 (2-yi)(i) \;dy
%+ \int_2^0 (x-2i)(-1) \;dx
%+ \int_2^0 (yi)(-i) \;dy \\
%&=
%  \int_0^2 x \;dx
%+ \int_0^2 2i+y \;dy 
%+ \int_0^2 x-2i \;dx
%+ \int_0^2 -y \;dy 
%\intertext{Canceling, by replacing $y$'s with $x$'s:}
%&=
%  \int_0^2 x \;dx
%+ \int_0^2 2i+x \;dx 
%+ \int_0^2 x-2i \;dx
%+ \int_0^2 -x \;dx \\
%&=
%  \int_0^2 x - x \;dx
%+ \int_0^2 (2i+x) + (x-2i) \;dx \\
%&= \int_0^2 2x \;dx \\
%&= \sq{x^2}_0^2 \\
%&= 4
%\end{align*}

From $0$ to $2$ ($S_1$):
The integral is $2$, since this reduces to a simple real integral $\int_0^2 x \;dx$.

From $2$ to $2+2i$ ($S_2$):
\begin{align*}
z(t) &= 2(1-t) + (2+2i)t \\
&= 2 - 2t + 2t + 2it \\
&= 2 + 2it \\
z'(t) &= 2i \\
\int_{S_2} f(z) \;dz
&= \int_0^1 f(z(t)) z'(t) \;dt \\
&= \int_0^1 \overline{2+2it} \cdot 2i \;dt \\
&= \int_0^1 (2-2it) \cdot 2i \;dt \\
&= \int_0^1 4i+4t \;dt \\
&= \int_0^1 4t + 4i \;dt \\
&= 4 \int_0^1 t \;dt + 4i \int_0^1 \;dt \\
&= 4 \sq{1/2} + 4i \sq{1}\\
&= 2+4i
\end{align*}

From $2+2i$ to $2i$ ($S_3$):
\begin{align*}
z(t) 
&= (2+2i)(1-t) + 2it \\
&= 2+2i-2t-2it+2it \\
&= 2+2i-2t \\
z'(t)
&= -2 \\
\int_{S_3} f(z) \;dz
&= \int_0^1 f(z(t)) z'(t) \;dt\\
&= \int_0^1 \overline{2+2i-2t} \cdot (-2) \;dt \\
&= -2 \int_0^1 (2-2t - 2i) \;dt \\
&= -2 \sq{(2-2i)\int_0^1 \;dt - 2\int_0^1 t\;dt}\\
&= -2 \sq{(2-2i) 1 - 2 (1/2)}\\
&= -2 \sq{2-2i - 1}\\
&= -2 \sq{1-2i}\\
&= -2 + 4i
\end{align*}

From $2i$ to $0$ ($S_4$):
\begin{align*}
z(t) &= 2i(1-t) + 0t \\
&= 2i-2it \\
z'(t) &= -2i \\
\int_{S_4} f(z) \;dz 
&= \int_0^1 f(z(t))z'(t) \;dt \\
&= \int_0^1 \overline{2i-2it} \cdot (-2i) \;dt \\
&= \int_0^1 (-2i+2it) \cdot (-2i) \;dt \\
&= \int_0^1 -4 + 4t \;dt \\
&= -4 \int_0^1 \;dt + 4 \int_0^1 t \;dt \\
&= -4 (1) + 4 (1/2) \\
&= -4 + 2 
\end{align*}
Thus,
\[ 
\int_S f(z)\;dz
= \sq{2} + \sq{2+4i} + \sq{-2+4i} + \sq{-4+2}
= \sq{2+2-2-4+2}+\sq{4i+4i}
= 8i
\]

\section{Problem 2}
The value of this integral over the spiral-shaped contour may be simplified. 
Namely, $z^2$ is entire; thus we may define a loop homotopy of $C$
into $C'(t)=1+t$ by
\begin{align*}
H &: [1,2] \times [0,1] \to \Co  &
H(t,s) &\overset{\text{def}}{=} (1-s) \cdot t e^{2 \pi i t} + s \cdot t
\end{align*}
Thus,
\begin{align*}
\int_C z^2 \;dz
&= \int_{C'} z^2 \;dz \\
&= \int_1^2 x^2 \;dx \\
&= \left.\frac{x^3}{3}\right|_1^2 \\
&= \frac{2^3}{3} - \frac{1^3}{3} \\
&= \frac{8-1}{3} \\
&= \frac{7}{3}
\end{align*}
\section{Problem 3}
\begin{align*}
\int_C \frac{\cos 2z}{(z-\pi)^3} \;dz
&= \frac{2 \pi i}{2!} \sq{D_z^2 (z \mapsto \cos(2z))}(\pi) \\
&= \pi i \sq{D_z (z \mapsto -2\sin(2z))}(\pi) \\
&= \pi i \sq{z \mapsto -4 \cos(2z)}(\pi) \\
&= -4 \pi i \cos(2\pi) \\
&= -4 \pi i
\end{align*}
\section{Problem 4}
By the maximum modulus principle, the greatest modulus for $e^z$ must
occur on the boundary. Therefore, we must only examine the three exposed edges 
the square as well as the circle boundary.

\[ 
\abs{e^z} 
= \abs{e^x(\cos y + i \sin y)}
= e^x (\cos^2 y + \sin^2 y)
= e^x
\]

Consequently, $\abs{e^z}=e^x$. Since this is monotonically increasing in $z$, 
exactly the point $z=1$ is the maximum, with value
$e$. Similarly, $\abs{e^z}$ is minimized on the entire edge $-2+[-1,1]i$.

\section{Problem 5}
Consider $g(z)=-if(z)=v(x,y) - iu(x,y)$, This is analytic by multiplication. Then 
suppose $\abs{v(x,y)} \le 10$ is bounded. Then $\exp \abs{v(x,y)} \le 10$.
But 
\[ 
10 
\ge \exp\abs{v(x,y)}
\ge \abs{\exp v(x,y)}
= \abs{\exp\p{v(x,y) - i u(x,y)}}
= \abs{\exp g(z)}
\]
For all $z \in \Co$. Thus, $\exp g(z)$, an entire analytic function, is bounded.
Thus it is constant by Liouville's Theorem. But this requires that $g(z)$ was
constant. However, $g$ is also injective; thus, 
$g$ must also be constant. Consequently, if $f(z) = u(x,y)+iv(x,y)$ is an entire
function and $\abs{v(x,y)} \le 10$, then $f$ is constant.

\section{Problem 6}
Since $D_z^{(n)} e^z = e^z$, and $e^{2 \pi i}=-1$, 
\[ a_n=-\frac{1}{n!}, \qquad n=0,1,2,\cdots \]
Thus, by Taylor's theorem, since $e^z$ is entire, we have the Taylor series
\[
e^z = \sum_{n=0}^\infty -\frac{1}{n!} (z-2 \pi i)^n
\]
which holds for all $z \in \Co$.

\section{Problem 7}
We assume we wish for the annulus $1 < \abs{z} < 5$.

The first portion, $\frac{1}{z+5}$, will yield Taylor series
(for $\abs{z}<5$),
\[ 
\frac{1}{z+5} 
= \frac{1}{5} \frac{1}{1 -(- z/5)}
= \frac{1}{5} \sum_{n=0}^\infty \p{\frac{z}{5}}^n
= \sum_{n=0}^\infty \frac{1}{5^{n+1}} z^n
\]

The second portion, $2/(z-i)$, has Laurent series (for $\abs{z} > 1$),
\[
\frac{2}{z-i} 
= \frac{2}{z} \frac{1}{1-i/z}
= \frac{2}{z} \sum_{n=0}^\infty \p{\frac{i}{z}}^n
= \frac{2}{z} \sum_{n=0}^\infty i^n z^{-n}
= \sum_{n=0}^\infty i^n z^{-n-1}
= \sum_{n=1}^\infty i^{n-1} z^{-n}
\]

By the uniqueness of Laurent series, 
and since the Taylor and Laurent series above are both correct on the annulus
$1 < \abs{z} < 5$, 
\[
f(z) 
= \frac{3z+10-i}{z^2+(5-i)z-5i}
= \frac{1}{z+5} + \frac{2}{z-i}
= \sum_{n=0}^\infty \frac{1}{5^{n+1}} z^n
+ \sum_{n=1}^\infty i^{n-1} z^{-n}, 
\qquad 1 < \abs{z} < 5
\]

\section{Problem 8}
\section{Problem 9}
\[ 
f(z)
= \frac{2}{z^3-6z^2+8z}
= \frac{A}{z} + \frac{B}{z-2} + \frac{C}{z-4}
\]
\[
2 
= A (z^2-6z+8) + B (z^2-4z) + C(z^2-2z)
= (A+B+C)z^2 + (-6A-4B-2C)z + 8A
\]
\[ A = 1/4 \]
\[ -6 (1/4) -4B -2C = 0 \]
\[ 2B+C=-3/4 \]
\[ C=-3/4 - 2B \]
\[ 0 = A+B+C = 1/4 + B -3/4-2B =-1/2-B\]
\[ B=-1/2 \]
\[ C=-3/4 - 2(-1/2) = -3/4 +1 = 1/4 \]
Thus,
\begin{align*}
A &= 1/4 & B &= -1/2 & C &= 1/4
\end{align*}
Thus, we know that
\[ 
f(z) 
= \frac{1/4}{z} + \frac{-1/2}{z-2} + \frac{1/4}{z-4} 
\]
Using Cauchy's integral formula, a counterclockwise loop around $0$, $2$, and $4$
yields...

...around $0$:
\[
\text{(This doesn't matter, so I didn't bother.)}
\]

...around $2$:
\[
\int_C \frac{\frac{2}{z(z-4)}}{(z-2)} \;dz 
= 2 \pi i \frac{2}{2(-2)} 
= 2 \pi i (-1/2)
= - \pi i
\]

...around $4$:
\[
\int_C \frac{\frac{2}{z(z-2)}}{(z-4)} \;dz
= 2 \pi i \frac{2}{4(4-2)} 
= 2 \pi i \frac{1}{2}
= \pi i
\]

\subsection{a}
This will be, using something pretty much like the Cauchy Residue Formula,
\[ -\pi i + \pi i = 0\]

\subsection{b}
Breaking this up appropriately, we add one of the twos and subtract a 4, giving us
\[ -\pi i - \pi i = -2\pi i \]

\subsection{c}
This can be broken up into two loops: one around 2 once, and one containing both 2
and 4. Thus, the integral is
\[ (-\pi i) + (-\pi i + \pi i) = - \pi i \]
\section{Problem 10}
\section{Problem 11}
\begin{theorem}
Suppose that $f$ is an entire function satisfying 
\[
f(z+i)=f(z) \text{ and } f(z+1)=f(z) \text{ for all } z
\]
Then $f$ is constant.
\end{theorem}
\begin{proof}
Note that if $f$ is entire it is continuous. Thus it takes compact subsets of $\Co$ to
compact subsets of $\Co$. Consequently, 
the image of $[0,1]+[0,1]i$ under $f$, $f([0,1] + [0,1]i)$, is bounded; that is,
\[
\abs{f([0,1]+[0,1]i)} < B
\]
for some bound $B \in \R$.

But then if $f([n,n+1] + [m,m+1]i)$ is bounded, we also know that by the presuppositions
of the theorem, all its neighboring cells share the same bound:
\begin{align*}
\abs{f([n,n+1]+[m,m+1]i}&=\abs{f([n+1,n+2]+[m,m+1]i} < B \\
\abs{f([n,n+1]+[m,m+1]i}&=\abs{f([n-1,n]+[m,m+1]i} < B \\
\abs{f([n,n+1]+[m,m+1]i}&=\abs{f([n,n+1]+[m+1,m+2]i} < B \\
\abs{f([n,n+1]+[m,m+1]i}&=\abs{f([n,n+1]+[m-1,m]i} < B \\
\end{align*}
Since every point of $\Co$ lies in some integer unit square in the complex
plane, and all of these by induction using the equations above are bounded
in modulus by $B$, $f$ is bounded in modulus by $B$ over all of $\Co$.
Since $f$ is bounded
in modulus by $B$, by the maximum modulus principle, $f$ must be constant.
\end{proof}

\end{document}
